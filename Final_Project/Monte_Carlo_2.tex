%%%%%%%%%%%%   LaTeX Preamble %%%%%%%%%%%%%%

\documentclass{beamer}

% list all packages
\usepackage{amsmath}
\usepackage{hyperref}
\usepackage{latexsym}
\usepackage{color}
\usepackage{mycommands}
\usepackage{listings}
\lstset{language=Fortran,
        keywordstyle=\color{blue}\textbf,
	commentstyle=\color[rgb]{0.133,0.545,0.133},
	stringstyle=\color[rgb]{0.627,0.126,0.941},
	breaklines=true,
	showstringspaces=false,
	frame=trBL
       }

% slide theme
\usetheme{Berlin}
\usecolortheme{mit}

% Set Logo
\pgfdeclareimage[height=0.5cm]{mit-logo}{../../mit-logo.pdf}
\logo{\vspace{-0.25cm}\pgfuseimage{mit-logo}\hspace*{0.025cm}}

% Include Custom environments
% % \setbeamertemplate{blocks}[rounded]
\setbeamertemplate{blocks}[rounded][shadow=true]

% \setbeamertemplate{headline}[default]
\setbeamertemplate{navigation symbols}{}

% \beamerdefaultoverlayspecification{<+->}

% Set color for 'alert' text
\setbeamercolor{alerted text}{fg=blue}

%\setbeamercolor{section in toc}

% Modify some default font sizes
\setbeamerfont{itemize/enumerate body}{size=\normalfont}
\setbeamerfont{itemize/enumerate subbody}{size=\smaller, shape=\upshape}
\setbeamerfont{frametitle}{size=\large, series=\bfseries}


% \setbeamertemplate{bibliography entry title}{}
% \setbeamertemplate{bibliography entry location}{}
% \setbeamertemplate{bibliography entry note}{}
\setbeamertemplate{bibliography item}[text] 

% \setbeamertemplate{items}[ball]
% \setbeamertemplate{itemize subitem}[circle-symbol]
% \setbeamertemplate{background canvas}[vertical shading][bottom=mitgray!25,top=white]

% Use the shrink option to squeeze lots of text on a slide

% \frame[shrink]{

% …

% }

\colorlet{dark green}{green!50!black}

\newcommand{\packin}{\setlength\abovedisplayskip{2pt}\setlength\belowdisplayskip{2pt}}

%\newcommand{\numberInBox}[2][0.9]%
%	{\scalebox{#1}{{\tikz \draw (0,0) node[refbox] {\makebox[\totalheight]{#2}};}}}

%\newcommand{\enumref}[2][0.9] {\numberInBox[#1]{\ref{#2}}}


% Small arrow pointing down and hooking right
\newcommand{\drarrow}{\scalebox{1.5}{\reflectbox{\rotatebox[c]{180}{$\boldsymbol{\smash[b]{\Rsh}}$}}}}

\newenvironment{prettydescript}[1]
	{\begin{list}{}%
		{\renewcommand\makelabel[1]{\itshape\bfseries\color{mitred} ##1:\hfill}%
		\settowidth\labelwidth{\makelabel{#1}}%
		\setlength\leftmargin{\labelwidth}%
		\addtolength\leftmargin{\labelsep}}}%
	{\end{list}}

\newenvironment{customdescript}[1]
	{\begin{list}{}%
		{\renewcommand\makelabel[1]{\bfseries\color{mitred} ##1\hfill}%
		\settowidth\labelwidth{\makelabel{#1}}%
		\setlength\leftmargin{\labelwidth}%
		\addtolength\leftmargin{\labelsep}}}%
	{\end{list}}

\makeatletter

\newenvironment{customlist}[2]{
  \ifnum\@itemdepth >2\relax\@toodeep\else
      \advance\@itemdepth\@ne%
      \beamer@computepref\@itemdepth%
      \usebeamerfont{itemize/enumerate \beameritemnestingprefix body}%
      \usebeamercolor[fg]{itemize/enumerate \beameritemnestingprefix body}%
      \usebeamertemplate{itemize/enumerate \beameritemnestingprefix body begin}%
      \begin{list}
        {
            \usebeamertemplate{itemize \beameritemnestingprefix item}
        }
        { \leftmargin=#1 \itemindent=#2
            \def\makelabel##1{%
              {%  
                  \hss\llap{{%
                    \usebeamerfont*{itemize \beameritemnestingprefix item}%
                        \usebeamercolor[fg]{itemize \beameritemnestingprefix item}##1}}%
              }%  
            }%  
        }
  \fi
}
{
  \end{list}
  \usebeamertemplate{itemize/enumerate \beameritemnestingprefix body end}%
}
\makeatother

\newenvironment<>{varblock}[2][\textwidth]{%
  \setlength{\textwidth}{#1}
  \begin{actionenv}#3%
    \def\insertblocktitle{#2}%
    \par%
    \usebeamertemplate{block begin}}
  {\par%
    \usebeamertemplate{block end}%
  \end{actionenv}}

%% Notational commands:
\newcommand{\params}{\ensuremath{\xi}}
\newcommand{\vparms}{\ensuremath{\gvect{\params}}}
\renewcommand{\thefootnote}{\ensuremath{\fnsymbol{footnote}}}
\setcounter{footnote}{2}
\renewcommand{\thempfootnote}{\ensuremath{\fnsymbol{mpfootnote}}}
\newcommand{\newsubsection}[1]{\subsection{#1}\setcounter{subsection}{0}}


% Title Page
\title[]{Monte Carlo Project Part 2}
\author[]{22.901 Introduction to Computer Programming for Nuclear Engineers}
\institute[\insertpagenumber]{}
\date{January 30 - February 2, 2012} 

% -----------------------------------------------------------------------------
\begin{document}
% -----------------------------------------------------------------------------

% Inset title page
\frame{\titlepage}
%------------------------------------------------------------------------------
\begin{frame}{Accesing pdfs}

  \begin{itemize}
    \item link in the pdf static library, already in Makefile
    \vfill\item function: get\_particle\_pos - send it slab length, it returns x location
    \vfill\item function: get\_particle\_mu - it returns mu (-1 to 1)
    \vfill\item function: get\_scatter\_mu - it returns mu(-1 to 1)
    \vfill\item function: get\_collision\_distance - send it the total xs and it returns the free-flight distance
    \vfill\item function: get\_collision\_type - send it, respectfully, the absorption, scattering and total xs and it returns the collision id
    \vfill\item ID=1 is Absorption and ID=2 is Scattering
  \end{itemize}

\end{frame}
%------------------------------------------------------------------------------
%------------------------------------------------------------------------------
\begin{frame}{Module \texttt{execute} - Subroutine \texttt{run\_problem}}

  \begin{itemize}
    \item call \texttt{initialize\_rng}
    \vfill\item begin loop over the number of histories
    \begin{itemize}
        \vfill\item call \texttt{particle\_init}
        \vfill\item call \texttt{reset\_tallies}
        \vfill\item get the slab id for the neutron, call \texttt{get\_slab\_id}
        \vfill\item begin loop \emph{while} the neutron is alive
	\begin{itemize}
	  \vfill\item call \texttt{transport}
	  \vfill\item call \texttt{interaction} \emph{if} the neutron is still alive
	\end{itemize}
    \end{itemize}
    \vfill\item call \texttt{bank\_tallies}
  \end{itemize}

\end{frame}
%------------------------------------------------------------------------------
\begin{frame}[allowframebreaks]{Module \texttt{execute} - Subroutine \texttt{transport}}

  \begin{itemize}
    \item declare 4 variables:
    \begin{itemize}
      \item real, \texttt{s} - free flight distance
      \item real, \texttt{newx} - temporary new x location
      \item real, \texttt{neig} - x location of nearest neighbor in direction
      \item logical, \texttt{resample} - resample distance
    \end{itemize}
    \vfill\item set resample to true
    \vfill\item begin a loop while resample is true
    \begin{itemize}
      \vfill\item access \texttt{get\_collision\_distance} function - see pdfs slide, set to free flight dist.
      \vfill\item based on free-flight and neutron mu compute new x location
      \vfill\item get nearest neighbor x in mu direction
      \vfill\item check for a surface crossing
	\vfill\item check if the neutron leaked, \alert{can only happen if slab\_id is 1 or n\_slabs}
	  \vfill\item kill neutron
	  \vfill\item set resample to false
	\vfill\item record track length to tally (only free-flight to surface)
	\vfill\item move neutron to neighbor surface
	\vfill\item change the slab number
      \vfill\item if the neutron did not cross a surface, it collided
      \vfill\item record full free-flight to track tally
      \vfill\item move neutron to new xlocation
      \vfill\item set resample to false

    \end{itemize}

  \end{itemize}

In summary: \\
Find a new flight distance. Project it on the x axis with its angle of travel. Determine the nearest
neighbor x location in the direction mu. If you cross a surface, then check if the neutron leaked out 
of a system.  Note: you can only leak if the neutron is originating in slab 1 or the final slab.  This can
be checked with the slab\_id. If you do leak, kill the particle and do not resample the distance.  If you cross
a surface even if you leak, move the neutron to the that surface.  Compute the free-flight distance to that surface
using the change in x and mu. Append that to the temporary tally.\\
If you don't cross a surface then you have a collision. Do not resample, tally the free-flight distance and exit the 
transport routine.
\end{frame}
%------------------------------------------------------------------------------
\begin{frame}{Module \texttt{execute} - Subroutine \texttt{interaction}}

  \begin{itemize}
    \item make a local integer variable called \texttt{id}
    \vfill\item add a 1/totalxs to the collision temp variable
    \vfill\item get the reaction type by accessing the function \texttt{get\_collision\_type} - see pdfs slide
    \vfill\item if the id is 1 then absorption happened - kill particle
    \vfill\item if the id is not 1 then scattering and sample a new angle
    \begin{itemize}
      \item to get a new angle access \texttt{get\_scatter\_mu} - see pdfs slide
    \end{itemize}
  \end{itemize}

\end{frame}
%------------------------------------------------------------------------------
\begin{frame}{Module \texttt{execute} - Subroutine \texttt{get\_slab\_id}}

  \begin{itemize}
    \item make a local integer variable called \texttt{id}
    \vfill\item add a 1/totalxs to the collision temp variable
    \vfill\item get the reaction type by accessing the function \texttt{get\_collision\_type} - see pdfs slide
    \vfill\item if the id is 1 then absorption happened - kill particle
    \vfill\item if the id is not 1 then scattering and sample a new angle
    \begin{itemize}
      \item to get a new angle access \texttt{get\_scatter\_mu} - see pdfs slide
    \end{itemize}
  \end{itemize}

\end{frame}
%------------------------------------------------------------------------------
\begin{frame}{Module \texttt{execute} - Subroutine \texttt{reset\_tallies}}

  \begin{itemize}
    \item loop around slabs
    \vfill\item call \texttt{tally\_reset} and send it a tally object for that slab
  \end{itemize}

\end{frame}
%------------------------------------------------------------------------------
\begin{frame}{Module \texttt{execute} - Subroutine \texttt{bank\_tallies}}

  \begin{itemize}
    \item loop around slabs
    \vfill\item call \texttt{bank\_tally} and send it a tally object for that slab
  \end{itemize}

\end{frame}
%------------------------------------------------------------------------------
\begin{frame}{Module \texttt{execute} - Subroutine \texttt{print\_tallies}}

  \begin{itemize}
    \item loop around slabs
    \vfill\item print out tally information by slab region 
    \vfill\item e.g. Slab 1  Coll: mean+/-var  Track: mean+/-var
  \end{itemize}

\end{frame}
%------------------------------------------------------------------------------
\begin{frame}{Module \texttt{global} - Subroutine \texttt{allocate\_problem}}

  \begin{itemize}
    \item no arguements just allocated the tally object for the number of slabs
  \end{itemize}

\end{frame}
%------------------------------------------------------------------------------
\begin{frame}{Module \texttt{global} - Subroutine \texttt{free\_memory}}

  \begin{itemize}
    \item no arguements just deallocate the tally object
  \end{itemize}

\end{frame}
%------------------------------------------------------------------------------
\begin{frame}{Module \texttt{particle} - Subroutine \texttt{particle\_init}}

  \begin{itemize}
    \item arguments are the particle object and length of slab
    \vfill\item call pdf function \texttt{get\_particle\_pos}
    \vfill\item call pdf function \texttt{get\_particle\_mu}
    \vfill\item make neutron alive
  \end{itemize}

\end{frame}
%------------------------------------------------------------------------------
\begin{frame}{Flux Volume Tallies}

  \begin{itemize}
    \item For each tally we have two varbles [s(c)1 and s(c)2] - used for mean and variance calculation
    \vfill\item We also have temp. variable track and coll
    \vfill\item During a neutrons life, events will be accumulated in track and coll
    \vfill\item When life over we ``bank'' them into s(c)1 and s(c)2
    \vfill\item Collision estimator (c)
\[
 \phi_{c} = \frac{1}{NX}\sum \frac{1}{\Sigma_{t}}
\]
    \item Track-length estimator (s)
\[
 \phi_{c} = \frac{1}{NX}\sum d
\] \\
\alert{$d$ is the free flight distance and I also use $s$ for this as well.}
  \end{itemize}

\end{frame}
%------------------------------------------------------------------------------
\begin{frame}{Module \texttt{tally} - Subroutine \texttt{tally\_reset}}

  \begin{itemize}
    \item set the track and coll variables in the tally instance to zero
  \end{itemize}

\end{frame}
%------------------------------------------------------------------------------
\begin{frame}{Module \texttt{tally} - Subroutine \texttt{bank\_tally}}

  \begin{itemize}
    \item Do the following math for the tally instance
  \end{itemize}
\[c1 = c1 + coll\] \\
\[  c2 = c2 + coll^{2} \] \\
\[  s1 = s1 + track \] \\
\[  s2 = s2 + track^{2} \]

\end{frame}
%------------------------------------------------------------------------------
\begin{frame}{Module \texttt{tally} - Subroutine \texttt{perform\_statistics}}
  \begin{itemize}
    \item pass in the tally instance, number of neutrons and volume
    \vfill\item compute mean for each tally estimator
\[
 mean = \frac{s(c)1}{dx*nhist}\] \\
\[
 var = \frac{1}{nhist - 1}\left[ \frac{s(c)2}{nhist} - \left( \frac{s(c)1}{nhist} \right)^{2} \right ]
\]

  \end{itemize}

\end{frame}
%------------------------------------------------------------------------------
\end{document}
