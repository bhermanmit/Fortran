%%%%%%%%%%%%   LaTeX Preamble %%%%%%%%%%%%%%

\documentclass{beamer}

% list all packages
\usepackage{amsmath}
\usepackage{hyperref}
\usepackage{latexsym}
\usepackage{color}
\usepackage{mycommands}
\usepackage{listings}
\lstset{language=Fortran,
        keywordstyle=\color{blue}\textbf,
	commentstyle=\color[rgb]{0.133,0.545,0.133},
	stringstyle=\color[rgb]{0.627,0.126,0.941},
	breaklines=true,
	showstringspaces=false,
	frame=trBL
       }

% slide theme
\usetheme{Berlin}
\usecolortheme{mit}

% Set Logo
\pgfdeclareimage[height=0.5cm]{mit-logo}{../../mit-logo.pdf}
\logo{\vspace{-0.25cm}\pgfuseimage{mit-logo}\hspace*{0.025cm}}

% Include Custom environments
% % \setbeamertemplate{blocks}[rounded]
\setbeamertemplate{blocks}[rounded][shadow=true]

% \setbeamertemplate{headline}[default]
\setbeamertemplate{navigation symbols}{}

% \beamerdefaultoverlayspecification{<+->}

% Set color for 'alert' text
\setbeamercolor{alerted text}{fg=blue}

%\setbeamercolor{section in toc}

% Modify some default font sizes
\setbeamerfont{itemize/enumerate body}{size=\normalfont}
\setbeamerfont{itemize/enumerate subbody}{size=\smaller, shape=\upshape}
\setbeamerfont{frametitle}{size=\large, series=\bfseries}


% \setbeamertemplate{bibliography entry title}{}
% \setbeamertemplate{bibliography entry location}{}
% \setbeamertemplate{bibliography entry note}{}
\setbeamertemplate{bibliography item}[text] 

% \setbeamertemplate{items}[ball]
% \setbeamertemplate{itemize subitem}[circle-symbol]
% \setbeamertemplate{background canvas}[vertical shading][bottom=mitgray!25,top=white]

% Use the shrink option to squeeze lots of text on a slide

% \frame[shrink]{

% …

% }

\colorlet{dark green}{green!50!black}

\newcommand{\packin}{\setlength\abovedisplayskip{2pt}\setlength\belowdisplayskip{2pt}}

%\newcommand{\numberInBox}[2][0.9]%
%	{\scalebox{#1}{{\tikz \draw (0,0) node[refbox] {\makebox[\totalheight]{#2}};}}}

%\newcommand{\enumref}[2][0.9] {\numberInBox[#1]{\ref{#2}}}


% Small arrow pointing down and hooking right
\newcommand{\drarrow}{\scalebox{1.5}{\reflectbox{\rotatebox[c]{180}{$\boldsymbol{\smash[b]{\Rsh}}$}}}}

\newenvironment{prettydescript}[1]
	{\begin{list}{}%
		{\renewcommand\makelabel[1]{\itshape\bfseries\color{mitred} ##1:\hfill}%
		\settowidth\labelwidth{\makelabel{#1}}%
		\setlength\leftmargin{\labelwidth}%
		\addtolength\leftmargin{\labelsep}}}%
	{\end{list}}

\newenvironment{customdescript}[1]
	{\begin{list}{}%
		{\renewcommand\makelabel[1]{\bfseries\color{mitred} ##1\hfill}%
		\settowidth\labelwidth{\makelabel{#1}}%
		\setlength\leftmargin{\labelwidth}%
		\addtolength\leftmargin{\labelsep}}}%
	{\end{list}}

\makeatletter

\newenvironment{customlist}[2]{
  \ifnum\@itemdepth >2\relax\@toodeep\else
      \advance\@itemdepth\@ne%
      \beamer@computepref\@itemdepth%
      \usebeamerfont{itemize/enumerate \beameritemnestingprefix body}%
      \usebeamercolor[fg]{itemize/enumerate \beameritemnestingprefix body}%
      \usebeamertemplate{itemize/enumerate \beameritemnestingprefix body begin}%
      \begin{list}
        {
            \usebeamertemplate{itemize \beameritemnestingprefix item}
        }
        { \leftmargin=#1 \itemindent=#2
            \def\makelabel##1{%
              {%  
                  \hss\llap{{%
                    \usebeamerfont*{itemize \beameritemnestingprefix item}%
                        \usebeamercolor[fg]{itemize \beameritemnestingprefix item}##1}}%
              }%  
            }%  
        }
  \fi
}
{
  \end{list}
  \usebeamertemplate{itemize/enumerate \beameritemnestingprefix body end}%
}
\makeatother

\newenvironment<>{varblock}[2][\textwidth]{%
  \setlength{\textwidth}{#1}
  \begin{actionenv}#3%
    \def\insertblocktitle{#2}%
    \par%
    \usebeamertemplate{block begin}}
  {\par%
    \usebeamertemplate{block end}%
  \end{actionenv}}

%% Notational commands:
\newcommand{\params}{\ensuremath{\xi}}
\newcommand{\vparms}{\ensuremath{\gvect{\params}}}
\renewcommand{\thefootnote}{\ensuremath{\fnsymbol{footnote}}}
\setcounter{footnote}{2}
\renewcommand{\thempfootnote}{\ensuremath{\fnsymbol{mpfootnote}}}
\newcommand{\newsubsection}[1]{\subsection{#1}\setcounter{subsection}{0}}


% Title Page
\title[]{Modules}
\author[]{22.901 Introduction to Computer Programming for Nuclear Engineers}
\institute[\insertpagenumber]{}
\date{January 30, 2012} 

% -----------------------------------------------------------------------------
\begin{document}
% -----------------------------------------------------------------------------

% Inset title page
\frame{\titlepage}

% Outline slide
\begin{frame}{Outline}
  \begin{itemize}
    \item Modules
  \end{itemize}
\end{frame}
%------------------------------------------------------------------------------
\begin{frame}{What are Modules for?}

  \begin{itemize}
    \item Modules fulfill multiple purposes
    \vfill\item Used for shared declarations (similar to headers)
    \vfill\item Used for defining global data
    \vfill\item Used for defining procedure interfaces
    \vfill\item Think of them as a high-level interface in your code
    \vfill
And more...
  \end{itemize}

\end{frame}
%------------------------------------------------------------------------------
\begin{frame}{Structure of a Module}

\texttt{module <name>} \\
  \hspace{0.1cm} \texttt{use definitions} \\
  \hspace{0.1cm} \texttt{implicit none} \\
  \hspace{0.1cm} \texttt{static data definitions, global to the module} \\
\texttt{contains} \\
  \hspace{0.1cm} \texttt{procedure definitions and interfaces} \\
\texttt{end module <name>}
  \vfill
  \begin{itemize}
    \item Keep 1 module to 1 file 
  \end{itemize}

\end{frame}
%------------------------------------------------------------------------------
\begin{frame}{How do modules interact?}

  \begin{itemize}
    \item Modules can \textbf{use} other modules
    \vfill\item Modules may not depend on themselves
    \vfill\item Two modules cannot depend on each other (circular dependency)
    \vfill\item Anything else will most likely work
  \end{itemize}
  \vfill
  \scriptsize
  For example: \\
  \texttt{module bicycle} \\
    \hspace{0.1cm}\texttt{implicit none}\\
    \hspace{0.1cm} \texttt{type :: wheel}\\
      \hspace{0.2cm}\texttt{integer :: spokes}\\
      \hspace{0.2cm}\texttt{real :: diameter}\\
      \hspace{0.2cm}\texttt{character(len=15) :: material}\\
    \hspace{0.1cm}\texttt{end type wheel}\\
  \texttt{end module bicycle}\\ 
  \vfill
  \texttt{program bikeshop}\\
    \hspace{0.1cm} \texttt{use bicycle}\\
    \hspace{0.1cm} \texttt{implicit none}

\end{frame}
%------------------------------------------------------------------------------
\begin{frame}{Global Data}

  \begin{itemize}
    \item  Variables in modules define global data
    \vfill\item  You need to specify the \textbf{save} attribute
    \vfill\item  If the whole routine is global data then can just specify one single save at the top under implicit none
    \vfill\item  Otherwise use a save attribute for each variable explicitly
    \vfill\item  Thus, if a variable is set in one procedure and used in another the save attribute is a must
  \end{itemize}

\end{frame}
%------------------------------------------------------------------------------
\begin{frame}{Procedures in Modules}

  \begin{itemize}
    \item Simplest to include procedures in modules 
    \item These procedures go after the \textbf{contains}
    \item Try not to use the same name for local variables in modules and the variables that are listed in the global scope in the file
    \item There isn't much more to it, just try and organize!
  \end{itemize}

\end{frame}
%------------------------------------------------------------------------------
\begin{frame}{Data and Procedure Accessibility}

  \begin{itemize}
    \item It is good practice to limit the scope of data and procedures in modules
    \item This is a very simple idea
    \item \textbf{private} names accessible only inside the module
    \item \textbf{public} names are accessible by the \textbf{use} command
    \item We will always do the following when creating a module:
  \end{itemize}
  \vfill
  \scriptsize
  \texttt{module somemodule} \\
    \hspace{0.1cm} \texttt{any global use commands inside module} \\
    \hspace{0.1cm} \texttt{implicit none} \\
    \hspace{0.1cm} \texttt{private} \\
    \hspace{0.1cm} \texttt{public ::  comma-separate list of public procedures} \\
    \hspace{0.1cm} \texttt{module global data (use public command as attribute)} \\
  \texttt{contains} \\
    \hspace{0.1cm} \texttt{procedures} \\
  \texttt{end module somemodule}

\end{frame}
%------------------------------------------------------------------------------
\begin{frame}{Partial Inclusion}

  \begin{itemize}
    \item When you state \textbf{use}, all public data and procedures can be accessed
    \vfill\item To partially include data or procedures use \textbf{only}
  \end{itemize}
  \vfill
  For example: \\
  \texttt{use bigmodule, only: errors, invert}
  \begin{itemize}
    \vfill\item  Makes only errors and invert visible
    \vfill\item  This is very good practice
    \vfill\item  Can easily find which procedures and data are used where (i.e. with grep)
    \vfill\item  You can only use - only \textbf{public} procedures from another module
    \vfill\item  This is why partial inclusion and accessiblilty are \\ good ideas
  \end{itemize}

\end{frame}
%------------------------------------------------------------------------------
\begin{frame}[allowframebreaks]{Module Circle}
\begin{scriptsize}
  \lstinputlisting{../examples/circle_test.f90}
  \lstinputlisting{../examples/class_Circle.f90}
\end{scriptsize}
\end{frame}
%------------------------------------------------------------------------------
\end{document}