%%%%%%%%%%%%   LaTeX Preamble %%%%%%%%%%%%%%

\documentclass{beamer}

% list all packages
\usepackage{amsmath}
\usepackage{hyperref}
\usepackage{latexsym}
\usepackage{color}
\usepackage{mycommands}
\usepackage{listings}
\lstset{language=Fortran,
        keywordstyle=\color{blue}\textbf,
	commentstyle=\color[rgb]{0.133,0.545,0.133},
	stringstyle=\color[rgb]{0.627,0.126,0.941},
	breaklines=true,
	showstringspaces=false,
	frame=trBL
       }

% slide theme
\usetheme{Berlin}
\usecolortheme{mit}

% Set Logo
\pgfdeclareimage[height=0.5cm]{mit-logo}{../../mit-logo.pdf}
\logo{\vspace{-0.25cm}\pgfuseimage{mit-logo}\hspace*{0.025cm}}

% Include Custom environments
% % \setbeamertemplate{blocks}[rounded]
\setbeamertemplate{blocks}[rounded][shadow=true]

% \setbeamertemplate{headline}[default]
\setbeamertemplate{navigation symbols}{}

% \beamerdefaultoverlayspecification{<+->}

% Set color for 'alert' text
\setbeamercolor{alerted text}{fg=blue}

%\setbeamercolor{section in toc}

% Modify some default font sizes
\setbeamerfont{itemize/enumerate body}{size=\normalfont}
\setbeamerfont{itemize/enumerate subbody}{size=\smaller, shape=\upshape}
\setbeamerfont{frametitle}{size=\large, series=\bfseries}


% \setbeamertemplate{bibliography entry title}{}
% \setbeamertemplate{bibliography entry location}{}
% \setbeamertemplate{bibliography entry note}{}
\setbeamertemplate{bibliography item}[text] 

% \setbeamertemplate{items}[ball]
% \setbeamertemplate{itemize subitem}[circle-symbol]
% \setbeamertemplate{background canvas}[vertical shading][bottom=mitgray!25,top=white]

% Use the shrink option to squeeze lots of text on a slide

% \frame[shrink]{

% …

% }

\colorlet{dark green}{green!50!black}

\newcommand{\packin}{\setlength\abovedisplayskip{2pt}\setlength\belowdisplayskip{2pt}}

%\newcommand{\numberInBox}[2][0.9]%
%	{\scalebox{#1}{{\tikz \draw (0,0) node[refbox] {\makebox[\totalheight]{#2}};}}}

%\newcommand{\enumref}[2][0.9] {\numberInBox[#1]{\ref{#2}}}


% Small arrow pointing down and hooking right
\newcommand{\drarrow}{\scalebox{1.5}{\reflectbox{\rotatebox[c]{180}{$\boldsymbol{\smash[b]{\Rsh}}$}}}}

\newenvironment{prettydescript}[1]
	{\begin{list}{}%
		{\renewcommand\makelabel[1]{\itshape\bfseries\color{mitred} ##1:\hfill}%
		\settowidth\labelwidth{\makelabel{#1}}%
		\setlength\leftmargin{\labelwidth}%
		\addtolength\leftmargin{\labelsep}}}%
	{\end{list}}

\newenvironment{customdescript}[1]
	{\begin{list}{}%
		{\renewcommand\makelabel[1]{\bfseries\color{mitred} ##1\hfill}%
		\settowidth\labelwidth{\makelabel{#1}}%
		\setlength\leftmargin{\labelwidth}%
		\addtolength\leftmargin{\labelsep}}}%
	{\end{list}}

\makeatletter

\newenvironment{customlist}[2]{
  \ifnum\@itemdepth >2\relax\@toodeep\else
      \advance\@itemdepth\@ne%
      \beamer@computepref\@itemdepth%
      \usebeamerfont{itemize/enumerate \beameritemnestingprefix body}%
      \usebeamercolor[fg]{itemize/enumerate \beameritemnestingprefix body}%
      \usebeamertemplate{itemize/enumerate \beameritemnestingprefix body begin}%
      \begin{list}
        {
            \usebeamertemplate{itemize \beameritemnestingprefix item}
        }
        { \leftmargin=#1 \itemindent=#2
            \def\makelabel##1{%
              {%  
                  \hss\llap{{%
                    \usebeamerfont*{itemize \beameritemnestingprefix item}%
                        \usebeamercolor[fg]{itemize \beameritemnestingprefix item}##1}}%
              }%  
            }%  
        }
  \fi
}
{
  \end{list}
  \usebeamertemplate{itemize/enumerate \beameritemnestingprefix body end}%
}
\makeatother

\newenvironment<>{varblock}[2][\textwidth]{%
  \setlength{\textwidth}{#1}
  \begin{actionenv}#3%
    \def\insertblocktitle{#2}%
    \par%
    \usebeamertemplate{block begin}}
  {\par%
    \usebeamertemplate{block end}%
  \end{actionenv}}

%% Notational commands:
\newcommand{\params}{\ensuremath{\xi}}
\newcommand{\vparms}{\ensuremath{\gvect{\params}}}
\renewcommand{\thefootnote}{\ensuremath{\fnsymbol{footnote}}}
\setcounter{footnote}{2}
\renewcommand{\thempfootnote}{\ensuremath{\fnsymbol{mpfootnote}}}
\newcommand{\newsubsection}[1]{\subsection{#1}\setcounter{subsection}{0}}


% Title Page
\title[]{Loops and Arrays}
\author[]{22.901 Introduction to Computer Programming for Nuclear Engineers}
\institute[\insertpagenumber]{}
\date{January 24, 2012} 

% -----------------------------------------------------------------------------
\begin{document}
% -----------------------------------------------------------------------------

% Inset title page
\frame{\titlepage}

% Outline slide
\begin{frame}{Outline}
  \begin{itemize}
    \item Control Constructs
    \item Arrays
  \end{itemize}
\end{frame}
%------------------------------------------------------------------------------
\begin{frame}{Named \texttt{IF} Statements}

  \begin{itemize}
    \item An \texttt{if} statement can be preceded by a name
    \vfill\item If this is used, the \texttt{end if} must be followed by the same name
    \vfill\item These are known as labels
  \end{itemize}
  \vfill
  \texttt{xcheck: if (x < 0.0) then} \\
    \hspace{0.1cm} \texttt{x = a} \\
  \texttt{else} \\
    \hspace{0.1cm} \texttt{x = c} \\
  \texttt{end if xcheck}

\end{frame}
%------------------------------------------------------------------------------
\begin{frame}{Select Case Construct}

  \begin{itemize}
    \item It is very common that there may be a lot of \texttt{if} statements for one variable
    \vfill\item Instead of putting a bunch of \texttt{else if} constructs, use \alert{select case} construct
  \end{itemize}
  \vfill
  \scriptsize{
  \texttt{print *, 'Happy Birthday, what is your age?'} \\
  \texttt{read *,age} \\
  \texttt{select case(age)} \\
  \texttt{case(18)} \\
    \hspace{0.1cm} \texttt{print *, 'You can now vote'} \\
  \texttt{case(40)} \\
    \hspace{0.1cm} \texttt{print *, 'Life begins'} \\
  \texttt{case(60)} \\
    \hspace{0.1cm} \texttt{print *, 'Free prescriptions'} \\
  \texttt{case default} \\
    \hspace{0.1cm} \texttt{print *, 'It's just another birthday!'} \\
  \texttt{end select}
  }
\end{frame}
%------------------------------------------------------------------------------
\begin{frame}{The \texttt{DO} Loop}

  \begin{itemize}
    \item A loop with some iteration counter, has the syntax: \\
      \vfill
      \hspace{1.9cm} \texttt{[loop name] do [loop control]} \\
      \hspace{2.5cm} \texttt{execution statements} \\
      \hspace{2cm} \texttt{end do [loop name]}
    \vfill\item \texttt{loop name} and \texttt{loop control} are optional
    \vfill\item If there is no loop control, it will go indefinitely (not good!)
  \end{itemize}

\end{frame}
%------------------------------------------------------------------------------
\begin{frame}{Loop Control}

  \begin{itemize}
    \item Simple Loop Control: \texttt{<ivar> = <LB>,<UB>}
    \vfill\item \texttt{<ivar>} \alert{must} be an integer variable
    \vfill\item \texttt{<LB>} and \texttt{<UB>} are the lower and upper bounds respectively
    \vfill\item \alert{Note}: The bounds can hold any integer expression
  \end{itemize}
  \vfill
  {\color{blue}{The flow of a loop::}} \\
  The variable starts at the lower bound \\
  If it exceeds the upper bound, the loop exits \\
  The loop body is executed \\
  The variable is incremented by \alert{ONE} \\
  The loop starts again

\end{frame}
%------------------------------------------------------------------------------
\begin{frame}{Specifying an Increment}

  \begin{itemize}
    \item The default increment step is +1 (will not decrement)
    \vfill\item To specify an increment: \texttt{<ivar> = <lb>,<ub>,<step>}
    \vfill\item Therefore, to perform a decreasing loop counter:
  \end{itemize}
  \vfill
  \texttt{history: do i = 20,1,-2} \\
    \hspace{0.1cm} \texttt{execution statements} \\
  \texttt{end do history}
\end{frame}
%------------------------------------------------------------------------------
\begin{frame}{Rules of \texttt{DO} Loops}

  \begin{itemize}
    \item Can't change bounds of loop once inside the loop
    \vfill\item \alert{Never} change the value of the loop variable
    \vfill\item Feel free to use the loop variables value anywhere \alert{inside} the loop
    \vfill\item Loop variable is not defined outside the context of a loop
    \vfill\item Do not use \texttt{GO TO} to exit a loop use \texttt{EXIT}
  \end{itemize}
  \vfill
  \texttt{exit} will break out of the nearest loop \\
  \texttt{cycle} will go increment the counter and go back to the top\\
  {\color{red}\begin{center}{Don't confuse \texttt{cycle} with \texttt{continue}}\end{center}}
\end{frame}
%------------------------------------------------------------------------------
\begin{frame}{While Loops}

  \begin{itemize}

    \item Useful when the a logical expression determines when the loop ends
    \vfill\item Has the syntax: \\
    \vfill
    \hspace{2cm}  \texttt{while (<logical expression>)} \\
      \hspace{2.5cm} \texttt{execution statements} \\
    \hspace{2cm} \texttt{end while}
    \vfill\item Loop will execute until \texttt{<logical expression>} is \texttt{.false.}
    \vfill\item I generally do not use these because they could go forever
    \vfill\item If a logical expression should terminate a loop I specify a do loop with a finite number of iterations
    \vfill\item I then use a conditional statement to \texttt{exit} the loop early

  \end{itemize}

\end{frame}
%------------------------------------------------------------------------------
\begin{frame}[allowframebreaks]{Nonlinear Bisection Algorithm}
\begin{scriptsize}
  \lstinputlisting{../examples/bisect/nonlinear.f90}
  \lstinputlisting{../examples/bisect/myfun.f90}
  \lstinputlisting{../examples/bisect/bisect.f90}
\end{scriptsize}
\end{frame}
%------------------------------------------------------------------------------
\begin{frame}{Declaring an Array - Terminology}

\texttt{real, dimension(10) :: a} \\
\texttt{real :: a(10),b(0:9,5,6:10)}\\

  \begin{itemize}

    \vfill
    \item The \emph{rank} is the number of dimensions
    \item \texttt{a} has rank 1 and \texttt{b} has rank 3
    \vfill
    \item The \emph{bounds} are the upper in lower limits
    \item If the range is not specified default is 1:dim
    \item \texttt{a} has bounds 1:10 and \texttt{b} has bounds 0:9,1:5,6:10
    \vfill
    \item A dimension's \emph{extent} is $UB-LB+1$
    \item \texttt{a} has extent 10 and \texttt{b} has extents 10,5,5

  \end{itemize}

\end{frame}
%------------------------------------------------------------------------------
\begin{frame}{Array Element Assignments}

  \begin{itemize}
    \item An array index can be any integer expression
    \vfill\item In order to declare size with a variable here, must use parameter
    \vfill\item Later in this lecture we will talk about \emph{allocation}
  \end{itemize}
  \vfill
  \texttt{integer,parameter :: n=50} \\
  \texttt{integer,dimension(-50:50) :: mark} \\
  \texttt{integer :: i}
  \texttt{do i = -n,n} \\
    \hspace{0.1cm} \texttt{mark(i) = 2*i} \\
  \texttt{end do}
  \vfill
  Sets mark to -100,-98,...,90,100
\end{frame}
%------------------------------------------------------------------------------
\begin{frame}{Array Element-wise Operations}

  \begin{itemize}
    \item Most built-in operators/functions are elemental
    \vfill\item They act element-by-element on arrays \\

    \vfill
    \hspace{1cm} \texttt{real,dimension(1:200) :: arr1,arr2,arr3} \\
    \hspace{1cm} \texttt{execution statements} \\
    \hspace{1cm} \texttt{arr1 = arr2 + 1.23*exp(arr3/4.56)} \\

    \vfill\item Comparisons and logical operations are also element-wise \\
    \vfill
    \hspace{1cm} \texttt{real,dimension(1:200) :: arr1,arr2,arr3} \\
    \hspace{1cm} \texttt{logical,dimension(1:200) :: flags} \\
    \hspace{1cm} \texttt{execution statements} \\
    \hspace{1cm} \texttt{flags = (arr1 > exp(arr2) .or. arr3 < 0.0)}
  \end{itemize}

\end{frame}
%------------------------------------------------------------------------------
\begin{frame}{Array Intrinsic functions}

  \begin{itemize}
    \item There are a lot of useful instrinsic procedures for arrays:
    \begin{itemize}
      \item \texttt{size(x[,n])} - the size of the nth dimension of an array
      \item \texttt{lbound(x[,n])} - the lower bound of the nth dimension of x
      \item \texttt{ubound(x[,n])} - the upper bound of the nth dimension of x
      \item \texttt{minval(x)} - the minimum of all elements of x
      \item \texttt{maxval(x)} - the maximum of all elements of x
      \item \texttt{sum(x[,n])} - the sum of all elements of x
      \item \texttt{product(x[,n])} - the product of all elements of x
      \item \texttt{transpose(x)} - the transpose of a matrix
      \item \texttt{dot\_product(x,y)} - the dot product of x and y
      \item \texttt{matmul(x,y)} - 1- and 2-D matrix multiplication
    \end{itemize}
    \item If you don't specify \texttt{n} and there is more than 1 dimension either, the operation is performed for all dimensions to give a scalar, or over each dimension resulting in a vector
  \end{itemize}

\end{frame}
%------------------------------------------------------------------------------
\begin{frame}{Element Ordering in Memory}

  \begin{itemize}
    \item The traditional term is ``column-major order''
    \vfill\item However, memory is only linear
    \vfill\item The easy way to remember is \alert{first index varies the fastest}
    \vfill\item Also means, the first index should be the inner-most nested loop (faster code!!)
    \vfill\item The elements are in the following order: 
  \end{itemize}
A(1,1),A(2,1),A(3,1),A(1,2),A(2,2),A(3,2),A(1,3),A(2,3),A(3,3)

{\begin{center}\color{red} Note: C, Matlab, etc. are the opposite\end{center}} 

\end{frame}
%------------------------------------------------------------------------------
\begin{frame}{Simple I/O of Arrays}

  \begin{itemize}
    \item Arrays can be included in I/O
    \vfill\item Remember \emph{column-major order}
  \end{itemize}
  \vfill
  \texttt{real, dimension(3,2) :: oxo} \\
  \texttt{read *,oxo} \\
  \begin{itemize}
    \vfill\item This is exactly equivalent to: 
  \end{itemize}
  \vfill
  \texttt{real :: oxo(3,2)} \\
  \texttt{read *, oxo(1,1), oxo(2,1), oxo(3,1), oxo(1,2), oxo(2,2), oxo(3,2)}

\end{frame}
%------------------------------------------------------------------------------
\begin{frame}{Array Constructors}

  \begin{itemize}
    \item An array constructor creates a temporary array and copies it to a variable \\
    \hspace{1cm} \texttt{integer :: marks(1:6)} \\
    \hspace{1cm} \texttt{marks = (/10,25,32,50,54,60/)}

    \vfill\item Can use variable expressions:
    \hspace{1cm} \texttt{(/x,2.0*y,sin(t*w/3.0),etc./)}

    \vfill\item A list can be set using an \alert{Implied-DO Loop}:
    \item Let $n=9$:
    \hspace{1cm} \texttt{(/0.0,(k/10.0,k=1,n),1.0/)}
    \item This lists 0.0 - 1.0 in increments of 0.1
  \end{itemize}

\end{frame}
%------------------------------------------------------------------------------
\begin{frame}{Allocatable Arrays}

  \begin{itemize}
    \item So far when we have created arrays, we have used \alert{static allocation}
    \vfill\item Allocatable arrays use \alert{dynamic allocation}
    \vfill\item A new attribute, \texttt{allocatable}, is used to declare an array with \emph{unknown} shape but \emph{known} rank! \\
      \hspace{0.1cm} \texttt{integer, dimension(:), allocatable :: counts} \\
      \hspace{0.1cm} \texttt{real, dimension(:,:,:), allocatable :: values}
    \vfill\item They become defined when the space is allocated: \\
      \hspace{0.1cm} \texttt{allocate(counts(1:1000))} \\
      \hspace{0.1cm} \texttt{allocate(value(0:N,-5:5,M:2*N+1))}
    \vfill\item When finished using, \alert{must} deallocate: \\
      \hspace{0.1cm} \texttt{deallocate(counts)} \\
      \hspace{0.1cm} \texttt{deallocate(values)}
  \end{itemize}

\end{frame}
%------------------------------------------------------------------------------
\begin{frame}{Cholesky Decomposition}

To solve $\mathbb{A} = \mathbb{L}\mathbb{L}^{T}$ , A must be symmetrix and positive definite: \\
\vfill
\begin{equation}
 \nonumber
  L_{jj} = \sqrt{A_{jj} - \sum_{k=1}^{j-1}{L_{jk}^{2}}}
\end{equation}
\vfill
\begin{equation}
 \nonumber
  \forall_{i>j}, L_{ij} = \left ( A_{ij} - \sum_{k=1}^{j-1} L_{ik}L_{jk} \right)/L_{jj}
\end{equation}


\end{frame}
%------------------------------------------------------------------------------
\begin{frame}[allowframebreaks]{Cholesky Algorithm}
\begin{scriptsize}
  \lstinputlisting{../examples/cholesky.f90}
\end{scriptsize}
\end{frame}
%------------------------------------------------------------------------------
\begin{frame}{End of Class/External Assignment}

To solve $\mathbb{C} = \mathbb{A}*\mathbb{B}$
\vfill
\begin{equation}
 \nonumber
  c_{ij} = \sum_{k=1}^{m} a_{ik}*b{kj}
\end{equation}
\vfill
A has dimensions n,m \\
B has dimensions p,q \\

Have user enter these and make sure you allocate \\
Check that m = p, if not terminate the code with a message \\
Call subroutine for matrixmultiplication \\
In subroutine, you will need a two do loops that are nested (one for column and one for row) \\
You will need intrinsic function sum \\
Fun this in standalone and use it in cholesky decomposition instead of \texttt{matmul} \\

\end{frame}
%------------------------------------------------------------------------------
\end{document}